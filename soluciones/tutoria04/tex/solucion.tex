
\subsection*{Ejercicio 3}

	Encuentre la serie de Laurent para:
	$$ f(z) = \dfrac{1}{z(z-4j)} $$
	%
	Centrada en el punto $z_0=2j$\par
	%
	%
	% SOLUCION
	\underline{Solución:}
	%
	Se aplica fracciones parciales:
	%
	$$ f(z) = \dfrac{A}{z} + \dfrac{B}{z-4j} $$
	%
	Se puede obterner que $A=\dfrac{1}{-4j}$ y $B=\dfrac{1}{4j}$ \par
	%
	Por lo que:
	$$ f(z) = -\dfrac{1}{4j} \left (\dfrac{1}{z} + \dfrac{1}{z-4j} \right) $$
	%
	Recordar que la serie de Taylor es:
	%
	$$ f(z) = \sum_{n=0}^{\infty} \dfrac{f^{(n)} (z_0)}{n!} (z-z_0)^n$$
	%
	Por lo tanto, la primer tarea es calcular las derivadas de $f(z)$ y evaluarlas en $z_0=2j$. \par
	\begin{align*}
		% Primera derivada:
		f^{(1)}(2j) &= -\dfrac{1}{4j} \left (-\dfrac{1}{(2j)^2} + \dfrac{1}{(-2j)^2} \right) = \dfrac{1}{4}\\
		% Segunda derivada:
		f^{(2)}(2j) &= -\dfrac{1}{4j} \left (\dfrac{1\cdot 2}{(2j)^3} - \dfrac{1\cdot 2}{(-2j)^3} \right) = 0\\
		% Tercera derivada:
		f^{(3)}(2j) &= -\dfrac{1}{4j} \left (-\dfrac{1\cdot 2 \cdot 3}{(2j)^4} + \dfrac{1\cdot 2 \cdot 3}{(-2j)^4} \right) = -\dfrac{1}{8}
	\end{align*}
	%
	Entonces se pueden notar varias transiciones:
	%
	\begin{itemize}
		\item En cada derivada consecutiva, los términos dentro del paréntesis cambian de signo.
		\item El numerador de los términos entre paréntesis es igual al factorial del número de derivada. Es decir, si la derivada es $n$, entonces en los numeradores se encuentra $n!$.
		\item El denominador de los términos entre paréntesis tiene potencia $n+1$, donde $n$ es el orden de la derivada.
	\end{itemize}
	%
	Por lo tanto, la $n-$ésima derivada de $f(z)$ evaluada en $z_0=2j$ es:
	%
	$$ f^{(n)}(2j) = -\dfrac{1}{4j}\left[ \dfrac{n!}{(2j)^{n+1}}(-1)^n - \dfrac{n!}{(-2j)^{n+1}} (-1)^n \right]$$
	%
	Que se puede simplificar a:
	$$ f^{(n)}(2j) = -\dfrac{(-1)^{n+1} n!}{2 (2j)^{n+2}} [1-(-1)^{n+1}] $$
	%
	Pero hay que recordar que los coeficientes $a_n$ de la serie de Taylor son:
	%
	$$ a_n = \dfrac{f^{(n)}(2j)}{n!} = -\dfrac{(-1)^{n+1}}{2 (2j)^{n+2}} [1-(-1)^{n+1}] $$
	%
	Aún hay un término que se puede simplificar! ¿Cuál y por qué?\par
	%
	Finalmente, la serie se puede denotar como:
	%
	$$ f(z) = \sum_{n=0}^{\infty} a_n (z-2j)^n, \qquad \text{para $|z-2j|<2$}$$
	%
	\underline{Recordatorio:}
	%
	también se puede calcular la serie a partir de división polinomial y es más sencillo. Inténtelo.
	
