% --- Ejercicio 1 -------- %
\begin{ejercicio}
    La función de variable compleja $f(z)$ tiene, entre otros, los siguientes desarrollos en serie de Laurent.\par
    \begin{enumerate}[a.]
        \item $f(z)=\sum\limits_{k=2}^{\infty}\dfrac{1}{2^k}(z+3)^k$\\
        \item $f(z)=\sum\limits_{k=-1}^{\infty}\dfrac{(-j)^{k+1}}{k(z-1)^k}$\\
        \item $f(z)=\sum\limits_{k=-4}^{\infty}2^k(z-1-j)^k$\\
        \item $f(z)=\sum\limits_{k=2}^{\infty}5^{-k}(z+j)^k$
    \end{enumerate}
    Para todas las series anteriores, se han utilizado regiones de convergencia que contienen como punto límite al punto donde ellas se centran.

    \begin{enumerate}
        \item Indique dónde al menos deben encontrarse polos, ceros (ambos con su respectivo orden), puntos regulares y singularidades.
        \item Indique el valor de los residuos de $f(z)$ en los cuatro puntos donde se centran las series anteriores.
    \end{enumerate}
\end{ejercicio}

% --- Ejercicio 2 --- %
\begin{ejercicio}
    Determine la localización y clasificación de las singularidades y los ceros de la función de variable compleja:
    $$ f(z)=\dfrac{1}{z^4-z^2(1+j)+j} $$
\end{ejercicio}

% --- Ejercicio 3 --- %
\begin{ejercicio}
    Determine los residuos para la función $f(z)=\dfrac{z+1}{z^2(1-z)}$
\end{ejercicio}

% --- Ejercicio 4 --- %
\begin{ejercicio}
    Esboce gráficamente las siguiente trayectorias, indicando su sentido, y además exprese la trayectoria con una ecuación no paramétrica (que no depende de $t$).\par
    \hspace{3pt}
    \begin{tabular}{lll}
        a. & $z(t)(-1+2j)t$ & para $1 \leq t \leq 2 $\\
        b. & $z(t)=2-jt$ & para  $-3 \leq t \leq 1$\\
        c. & $z(t)=1+j+e^{-j\pi t}$ & para $0 \leq t \leq 1$\\
        d. & $z(t)=min(t+1;2)+j max(t-2;-1)$ & para $0 \leq t \leq 4$
    \end{tabular}
\end{ejercicio}

% --- Ejercicio 5 --- %
\begin{ejercicio}
    Esboce gráficamente y represente de forma paramétrica las siguientes trayectorias con $0 \leq t \leq 1$:
    \begin{enumerate}[a.]
        \item Segmento de recta de $1-j$ a $2+2j$.
        \item Círculo unitario en sentido horario.
        \item $|z-1+2j|=2$ en sentido antihorario.
    \end{enumerate}
\end{ejercicio}

% --- Ejercicio 6 --- %
\begin{ejercicio}
    Encuentre el valor de las integrales:
    \begin{enumerate}[a.]
        \item $\displaystyle\int_C (x^2+j2xy+y^2)dz$
        \item $\displaystyle\int_C z^2 dz$
    \end{enumerate}
    Para las trayectorias de integración de los puntos 4.d y 5.a.
\end{ejercicio}

% --- Ejercicio 7 --- %
\begin{ejercicio}
    Evalúe las siguientes integrales:
    \begin{enumerate}[a.]
        \item $\displaystyle\int_C z^2 dz$
        \item $\displaystyle\int_C (x^2+y^2) dz$
    \end{enumerate}
    Para los contornos:
    \begin{enumerate}
        \item Segmento de recta de $1$ a $j$.
        \item Segmento de círculo $|z|=1$ que va en sentido positivo de $1$ a $j$
    \end{enumerate}
\end{ejercicio}
