% --- Ejercicio 1 -------- %
\begin{ejercicio}
    Sea la función:
    $$ f(z)=\frac{\cos(z+1)}{z(z+1)(z-1)(z^2-6z+25)} $$
    Inidique cuántos posibles desarrollos de Laurent centrados en $z_0=3$ existen para $f(z)$ y las correspondientes regiones de convergencia.
\end{ejercicio}

% --- Ejercicio 2 -------- %
\begin{ejercicio}
    Encuentre la representación en series de potencias de la función:
    $$ f(z)=\frac{1}{z-j} $$
    En las regiones:
    \begin{enumerate}
        \item $|z|<1$
        \item $|z|>1$
        \item $1<|z-1-j|<\sqrt{2}$
    \end{enumerate}
\end{ejercicio}

% --- Ejercicio 3 -------- %
\begin{ejercicio}
    Encuentre el desarrollo en serie de Taylor para la siguiente función:
    $$ \frac{1}{z(z-4j)} $$
    Centrado en el punto $z_0=2j$
\end{ejercicio}

% --- Ejercicio 4 -------- %
\begin{ejercicio}
    Encuentre la serie de Laurent para:
    $$ f(z)= \frac{1}{z(z-1)^2} $$
    Alrededor de $z_0=0$ y $z_0=1$, especifique las posibles regiones de convergencia para cada caso.
\end{ejercicio}

% --- Ejercicio 5 -------- %
\begin{ejercicio}
    Encuentre la serie de Laurent para:
    $$ f(z) = \frac{1}{(z-1)(z+2)} $$
    Centrada alrededor de $z_0=-1$ para una región de convergencia anular.
\end{ejercicio}

% --- Ejercicio 6 -------- %
\begin{ejercicio}
    Determine la expansión en serie de Laurent de $f(z)=z^2 \sin (\frac{1}{z})$ alrededor de $z_0=0$.
\end{ejercicio}

\def\arraystretch{2}%
\newcommand{\midmatch}{\hspace{0.75in}\underline{\hspace{0.5in}     }}
% --- Ejercicio 7 -------- %
\begin{ejercicio}
    La columna izquierda contiene cuatro expansiones en serie de Laurent para la función $f(z) = \frac{z}{(z-1)(2-z)}$. En la columna de la derecha se muestran las regiones de convergencia para los desarrollos de las series propuestas. Asocie cada una de las expansiones con su región de convergencia correspondiente.
    \begin{table}[!h]
        \centering
        \begin{tabular}{llll}

            & & \midmatch & $1<|z|<2$\\
            %\hline
            a. & $f(z) = \dfrac{1}{2}z+\dfrac{3}{4}z^2+\dfrac{7}{8}z^3+\dfrac{15}{16}z^4+\cdots$ & \midmatch & $|z-1|>2$\\
            %\hline
            b. & $f(z) = \cdots -\dfrac{1}{z^2}-\dfrac{1}{z}-1-\dfrac{z}{2}-\dfrac{z^2}{4}- \dfrac{z^3}{8}+\cdots$ & \midmatch & $|z-2|>2$\\
            %\hline
            c. & $f(z) = \dfrac{1}{z-1}+\dfrac{2}{(z-1)^2}+\dfrac{2}{(z-1)^3}+\cdots$ & \midmatch & $|z|<1$\\
            %\hline
            d. & $f(z) = \dfrac{2}{z-2}-1+(z-2)-(z-2)^2+(z-2)^3-\cdots$ & \midmatch & $0<|z-2|<1$\\
            %\hline
            & & \midmatch & $|z-1|<1$\\

        \end{tabular}

    \end{table}
\end{ejercicio}

% --- Ejercicio 8 --- %

\begin{ejercicio}
    Se sabe que una función $f(z)$ se puede expandir en una serie de potencias centrada en $z_0=1$ de la forma:
    $$ f(z) = \sum_{n=0}^{\infty}a_n(z-1)^n $$

    Para todo $z$ dentro de la región de convergencia $|z-1|<\frac{1}{2}$.\par
    Indique cuál región de convergencia tiene la siguiente serie:
    $$ f(z) = \sum_{n=-\infty}^{\infty}a_n \frac{z+1}{2(z-1)^n} $$
    Si los coeficientes $a_n$ son los mismos en ambas series.
\end{ejercicio}
