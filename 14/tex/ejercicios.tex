\begin{ejercicio}
	Encuentre la transformada $z$ de
	$$ x(n) = \frac{u(n-2)}{4^n} $$
	Con su respectiva ROC.
\end{ejercicio}

\begin{ejercicio}
	Encuentre la transformada $z$ de 
	$$ x(n) = \left(\frac{1}{2}\right)^n \cos \left( \frac{n\pi}{4} \right) u(-n) $$
	Indique los polos, ceros y su respectiva ROC. ¿Es absolutamente sumable $x(n)$?
\end{ejercicio}

\begin{ejercicio}
	Encuentre la transformada $z$ inversa de:
	$$ X(z) = \cos (z) $$
	Sabiendo que el círculo unitario del plano $z$ se encuentra en la ROC.
\end{ejercicio}

\begin{ejercicio}
	Encuentre, por división polinomial, la transformada $z$ inversa de
	$$ X(z) = \frac{1+z^{-1}}{1+\frac{1}{3}z^{-1}} $$
	Para las ROC: $|z|>\frac{1}{3}$ y $|z|< \frac{1}{3}$.
\end{ejercicio}

\begin{ejercicio}
	Encuentre la transformada $z$ inversa de
	$$ X(z) = \frac{1-\frac{1}{3}z^{-1}}{(1-z^{-1})(1+2z^{-1})} $$
	Para cada una de las ROC posibles.
\end{ejercicio}

\begin{ejercicio}
	Encuentre la transformada $z$ inversa de:
	$$ X(z) = \frac{1}{256}\left[\dfrac{256-z^{-8}}{1-\frac{1}{2}z^{-1}} \right], \qquad ROC:|z|>0 $$
\end{ejercicio}

\begin{ejercicio}
	Determine la convolución $x_1(n)*x_2(n)*x_3(n)$ si se sabe que
	\begin{align*}
		x_1(n) &= \{ 4,3,2,\underset{\uparrow}{1} \} \\
		x_2(n) &= \{ -1,0,\underset{\uparrow}{1},0,-1 \} \\
		x_3(n) &= \delta(n+2)
	\end{align*}
\end{ejercicio}

\begin{ejercicio}
	Un sistema LTI tiene función de transferencia $H(z)$ y respuesta al impulso $h(n)$. Si se sabe que:
	\begin{enumerate}[a.]
		\item $h(n)$ es real.
		\item $h(n)$ es derecha.
		\item $\lim_{z\to\infty} H(z)=1$
		\item $H(z)$ tiene dos ceros.
		\item $H(z)$ tiene dos polos $z_1$ y $z_2$.
		\item $|z_1| = \frac{3}{4}$ 
	\end{enumerate}
	¿El sistema es estable? ¿Es causal?
\end{ejercicio}