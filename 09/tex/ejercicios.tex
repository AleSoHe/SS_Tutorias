% --- Ejercicio 1 --- %

\begin{ejercicio}
    Determine la transformada de Fourier de $x(t)=1+\sin(2\pi t-\pi/4)$
\end{ejercicio}

\begin{ejercicio}
    Encuentre la señal $x(t)$ que tiene como transformada de Fourier a $X(j\omega) = 2\pi \sigma (\omega) + \pi \sigma (\omega-4\pi) - \pi \sigma (\omega + 4\pi)$
\end{ejercicio}

\begin{ejercicio}
    Asocie a cada función no periódica en el tiempo mostrada al lado izquierdo su correspondiente espectro, dado a través de sus partes real e imaginaria. Para esto, utilice las propiedades de la Transformada de Fourier.\\[15pt]
    %
    \begin{minipage}{5in}
        \begin{tabular}{|c|l|}
            \hline
            A & \includegraphics[width=0.35\textwidth]{figs/time1.pdf}\\
            \hline
            B & \includegraphics[width=0.35\textwidth]{figs/time2.pdf}\\
            \hline
            C & \includegraphics[width=0.35\textwidth]{figs/time3.pdf}\\
            \hline
            D & \includegraphics[width=0.35\textwidth]{figs/time4.pdf}\\
            \hline
        \end{tabular}
    \hspace{0.4cm}
        \begin{tabular}{|c|c|c|}
            \hline
                (  ) &
                \includegraphics[width=0.35\textwidth]{figs/re1.pdf}
                \includegraphics[width=0.35\textwidth]{figs/im1.pdf}
            \\
            \hline
                (  ) &
                \includegraphics[width=0.35\textwidth]{figs/re2.pdf}
                \includegraphics[width=0.35\textwidth]{figs/im2.pdf}
            \\
            \hline
                (  ) &
                \includegraphics[width=0.35\textwidth]{figs/re3.pdf}
                \includegraphics[width=0.35\textwidth]{figs/im3.pdf}
            \\
            \hline
                (  ) &
                \includegraphics[width=0.35\textwidth]{figs/re4.pdf}
                \includegraphics[width=0.35\textwidth]{figs/im4.pdf}
            \\
            \hline
        \end{tabular}
    \end{minipage}
\end{ejercicio}

\begin{ejercicio}
    Sea $x(t)$ una función que se puede expresar como la resta $x(t)=h(t)-h(-t)$. Donde $h(t)$ es una función de valor real. Si la transformada de Fourier de $h(t)$ es $H(j\omega)$ y la de $x(t)$ es $X(j\omega)$, entonces utilice las propiedades de la transformada de Fourier para encontrar $X(j\omega)$ en términos de la parte imaginaria de $H(j\omega)$.
\end{ejercicio}

\begin{ejercicio}
    \label{ej:triang}
    Encuentre la transformada de Fourier de la función de la figura \ref{fig:triang}, utilizando linealidad y la propiedad de derivación, exprese, en caso de ser posible, el resultado en términos puramente reales o puramente imaginarios.
    \begin{figure}[!h]
        \centering
        \includestandalone{figs/triang}
        \caption{Función a utilizar en el ejercicio \ref{ej:triang}.}
        \label{fig:triang}
    \end{figure}
\end{ejercicio}

\begin{ejercicio}
    \label{ej:mountain}
    Considerando que $u(t)$ es el escalón unitario, demuestre que las transformaciones de Fourier de las siguientes funciones son:
    \begin{center}
        \begin{tabular}{lll}
            $x_1(t) = \left\{ \begin{array}{ll}
                \sin(t) & -\frac{\pi}{2} \leq t \leq \frac{\pi}{2} \\
                0       & \text{en el resto}
            \end{array}
            \right.$
            & \laplace & $X_1(j\omega) = j2\omega \dfrac{\cos(\frac{\pi}{2}\omega)}{\omega ^2-1}$\\[25pt]
            %
            $r(t) = u(t+\frac{1}{2})-u(t-\frac{1}{2})$ & \laplace &
             $R(j\omega) = sa(\frac{\omega}{2})$
        \end{tabular}
    \end{center}
    En este problema usted deberá utilizar las propiedades de la transformada de Fourier para encontrar, a partir de las funciones anteriores, la transformada de otra función $f(t)$ más compleja, mostrada en la figura \ref{fig:mountain}.
    \begin{figure}
        \centering
        \includestandalone[width=0.5\textwidth]{figs/mountain}
        \caption{Función $f(t)$ a utilizar en el ejercicio \ref{ej:mountain}}
        \label{fig:mountain}
    \end{figure}
    \begin{enumerate}
        \item Defínase ahora la función:
        $$x_2(t) = \left\{
            \begin{array}{ll}
                f(t) & -\infty < t \leq -\frac{1}{2}\\
                0    & \text{en el resto}
            \end{array}
            \right.
        $$
        Esta función $x_2(t)$ puede obtenerse también como una combinación de las funciones $x_1(t)$ y $r(t)$ especificadas en el enunciado, tal que:
        $$ x_2(t) = \alpha x_1(\beta t-\tau_0) + k r(\gamma t - \tau_1) $$
        Encuentre los valores de $\alpha, \beta, k, \gamma, \tau_0$ y $\tau_1$ que cumplen con esa tarea. (Sugerencia: realice los desplazamientos temporales como última operación; es decir, encuentre primero una función idéntica a la buscada excepto por su posición y luego realice el desplazamiento adecuado).

        \item Si para el intervalo $t\in \left[\frac{1}{2},\infty\right]$ se define $x_3(t)=f(t)$ y fuera de ese intervalo $x_3(t) = 0$, entonces encuentre una expresión para $x_3(t)$ primero en términos de $x_2(t)$ y luego en términos de $x_1(t)$.
        \item Encuentre una expresión para $f(t)$ en términos de $r(t)$, $x_2(t)$ y $x_3(t)$.
        \item Encuentre la transformada de Fourier de $x_2(t)$ en térmnios de $X_1(j\omega)$ y $R(j\omega)$.
        \item Encuentre la transformada de Fourier de $x_3(t)$ en térmnios de $X_2(j\omega)$.
        \item ¿$X_1(j\omega)$ es una función par o impar? Justifique.
        \item ¿$R(j\omega)$ es una función par o impar? Justifique.
        \item Encuentre la transformada de Fourier de $f(t)$ utilizando los resultados anteriores. Considere la simetría de $f(t)$ y exprese el resultado en términos únicamente reales o imaginarios, según corresponda.
    \end{enumerate}
\end{ejercicio}
