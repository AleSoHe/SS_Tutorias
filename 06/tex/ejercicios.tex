% --- Ejercicio 1 -------- %
\begin{ejercicio}
    Evalúe la siguiente integral considerando el contorno de integración\\ $C:|z+3|=1$
    $$ \oint_C \dfrac{1}{(z^2-1)(z+3)} \,dz$$
\end{ejercicio}

% --- Ejercicio 2 -------- %
\begin{ejercicio}
    Si $C$ es el semicírculo en sentido positivo conformado por los puntos de frontera $z \in \mathbb{C}$, de la región $|z|\leq 2$, $Re\{z\}>0$, entonces determine el resultado de la siguiente integral comlpleja
    $$ \int_C \dfrac{\sin(z\frac{\pi}{2})}{1-z^3}\,dz $$
\end{ejercicio}

% --- Ejercicio 3 -------- %
\begin{ejercicio}
    Evalúe la integral $\displaystyle \oint_C z^* \,dz$ cuando el contorno de integración $C$ es $|z|=2$.
\end{ejercicio}

% --- Ejercicio 4 -------- %
\begin{ejercicio}
    Evalúe las siguiente integrales reales utilizando métodos de integración compleja
    $$ \int_{-\infty}^{\infty} \dfrac{1}{(x^2+4x+5)^2} \,dx $$
    $$ \int_{-\infty}^{\infty} \dfrac{1}{(x^2+4)^3} \,dx $$
\end{ejercicio}

% --- Ejercicio 5 -------- %
\begin{ejercicio}
    Resuelva las siguientes integrales trigonométricas por medio de integración comlpleja
    $$ \int_0^{2\pi} \dfrac{\cos(\theta)}{3+2\cos(\theta)} \,d\theta $$
    $$ \int_0^{\pi}  \dfrac{\,d\theta}{2+\sin^2(\theta)} $$
\end{ejercicio}
